%\documentstyle[12pt,vita,fullpage]{article}
\documentclass[12pt]{article}

\usepackage{vita}
\usepackage{myfullpage}

\setlength{\parindent}{0em}

\title{Curriculum Vita}
\author{Wayne O. Cochran\\
       Assistant Professor, Computer Science\\
       Washington State University Vancouver\\
       14204 NE Salmon Creek Avenue\\
       Vancouver, WA 98686-9600\\
       (360) 546-9463\\
       {\tt cochran@vancouver.wsu.edu}}
%       {\tt http://www.eecs.wsu.edu/\~{}wcochran}}

\begin{document}

\maketitle

\noindent{\large\bf Current Appointment}\\

Assistant Professor in the the School of Electrical Engineering and 
Computer Science at Washington State University Vancouver. 
Appointment began
in August, 1999, as tenure track faculty.

\vspace{0.5cm}

\noindent{\large\bf Research Interest}\\

Computer Graphics, Geometric Modeling, and Real Time Shading.
Current work involves modeling and rendering intricate surfaces
%and terrain 
using fractal interpolation surfaces, and real-time illumination
using texture maps.

\vspace{0.5cm}

%\noindent{\large\bf Education}\\[2ex]
\noindent{\large\bf Education}

\begin{itemize}
\item {\bf Ph.D.} Computer Science, 1998,
Washington State University. \\
%School of Electrical Engineering and Computer Science.
Dissertation Title: ``A Recurrent Modeling Toolset.''
%Advisor: John C. Hart.
%GPA: 3.98. 

\item {\bf M.S.} Computer Science, 1994,
Washington State University. \\
%School of Electrical Engineering and Computer Science.
Thesis title: ``Fractal Volume Compression.''
%Advisor: John C. Hart.
%GPA: 4.00.

\item {\bf B.S.} Mathematics, cum laude, 1990.
% College of Arts and Sciences,
University of Washington. 
%GPA: 3.75.
\end{itemize}

\vspace{0.25cm}

%\pagebreak

%\begin{itemize}
%\item Computer Graphics
%  \begin{itemize}
%  \item Rendering  
%  \end{itemize}
%\item Geometric Modeling
%  \begin{itemize}
%  \item Modeling Complex Geometry
%  \end{itemize}
%\item Scientific Visualization
%  \begin{itemize}
%  \item Volume Visualization
%  \item Volume Data Compression
%  \end{itemize}
%\end{itemize}

\noindent{\large\bf Professional Experience}

\begin{itemize}

\item {\bf Assistant Professor}, 1999-present,
Washington State University Vancouver.
Current research area involves intricate surface modeling
using recurrent models, and real-time illumination of
relief surfaces using texture maps.
Courses taught at WSUV:
{\em Automata and Formal Languages, Operating Systems and
Computer Architecture, Introduction to Computer Graphics,
Advanced Computer Graphics} and {\em Programming Language Design.}
Heavily involved in program development for 
computer science program which started in the fall of 1999.

\item {\bf Instructor}, 1998-1999,
Washington State University. Performed research and taught
courses in Introduction to Microprocessors, C Programming,
Windows Programming, Computer Networks, and Java Programming.

\item {\bf Research Assistant}, 1994-1998,
Washington State University.
Research included volumetric compression using fractals,
and other techniques using fractals to model complex geometry
(e.g. rough curves and surfaces). Research was part of the
{\em Recurrent Modeling Project} funded by Intel and a grant
from the NSF.
Also acted as substitute teacher for Computer Graphics, 
Advanced Computer Graphics, and Numerical Computing. 

\item {\bf Research/Teaching Assistant}, 1992--1994, 
Washington State University.
Contributed research for and implementation of knowledge based systems.
Lab instructor for
introductory computer programming courses.

\item {\bf Software Engineer}, 1990-1992,
Raster Graphics Inc
(assets acquired by Peritek Corporation in 2001)
{\tt http://www.rastergraf.com}
1804-P SE First St.
Redmond, Oregon 97756
% 1804 SE First Street
%Redmond, OR 97756.
(541) 923-5530
Job Overview: 
Design and implementation of rasterization firmware for
industrial graphics boards. Design and implementation of
3D rendering libraries.
\end{itemize}

\vspace{0.25cm}

%\pagebreak

\noindent{\large\bf Publications}

\begin{enumerate}

\item Wayne O. Cochran,
Recurrent Interpolation Surfaces,
{\em Proceedings of the Western Computer Graphics Symposium,}
March 2003, pp.~9--15.

\item W. O. Cochran, R. R. Lewis, J. C. Hart,
The Normal of a Fractal Surface,
{\em The Visual Computer,} vol. 17, no. 4, April 2001, pp.~209--218.

\item Wayne O. Cochran,
Fractal Interpolation Surfaces for Digital Elevation Maps,
{\em Proceedings of the Western Computer Graphics Symposium,}
March 2001, pp.~8--14.

\item Wayne O. Cochran,
A Recurrent Modeling Toolset, Ph.D. dissertation.
Washington State University, December 1998.

\item
Wayne O. Cochran, John C. Hart, Patrick J. Flynn,
On Approximating Rough Curves with Fractal Functions,
{\em Proceedings of Graphics Interface,}
June 1998.

\item
J.C. Hart, P.J. Flynn, W.O. Cochran. 
Similarity Hashing: A Model-Based Vision Solution to the Inverse 
Problem of Recurrent Iterated Function Systems. 
{\em Fractals 5} April 1997, pp.~39-50.

\item
Wayne O. Cochran, John C. Hart, Patrick J. Flynn,
Hashing Fractal Functions
{\em Proceedings of the Western Computer Graphics Symposium,}
April 1997, pp.~69--78.

\item
Wayne O. Cochran, John C. Hart, Patrick J. Flynn,
Fractal Volume Compression,
{\em IEEE Transactions on Visualization and Computer Graphics}
2 (4),
%vol. 2, no. 4, 
December 1996, pp.~313--322.

\item
Wayne O. Cochran, John C. Hart, Patrick J. Flynn,
Similarity and Affinity Hashing,
{\em Proceedings of the Western Computer Graphics Symposium,}
March 1996, pp.~89--100.

\item
W.O. Cochran, J.C. Hart and P.J. Flynn. Recurrent Modeling. 
{\em Intel Forum: Enabling Live Media in Cyberspace,} 
invited poster. January 1996.

\item
Wayne O. Cochran, John C. Hart, Patrick J. Flynn,
Principal Component Classification for Fractal Volume Compression,
{\em Proceedings of the Western Computer Graphics Symposium,}
March 1995, \mbox{pp.~9--18}.

\end{enumerate}

%\vspace{0.25cm}
%\pagebreak

\noindent{\large\bf Courses Taught}

\begin{enumerate}
\item Fall 2003, CptS 442/542, Computer Graphics, 3 credits, 40 students.
\item Fall 2003, CptS 317, Automata and Formal Languages, 3 credits,
   20 students.
\item Spring 2003, CptS 355, Programming Language Design, 3 credits, 
    26 students
\item Spring 2003, CptS 548, Advanced Computer Graphics, 3 credits, 2 students
\item Fall 2002, CptS 442/542, Computer Graphics, 3 credits, 12 students.
\item Fall 2002, CptS 317, Automata and Formal Languages, 3 credits, 
  26 students
\item Spring 2002, CptS 355, Programming Language Design, 3 credits, 
    13 students
\item Fall 2001, CptS 442/542, Computer Graphics, 3 credits, 40 students.
\item Spring 2001, CptS 460, Operating Systems, 3 credits, 10 students.
\item Fall 2000, CptS 317, Automata and Formal Languages, 3 credits, 
   10 students.
\item Fall 2000, CptS 499, Special Problems, 2 credits, 5 students.
\item Spring 2000, CptS 460, Operating Systems, 3 credits, 12 students.
\item Spring 2000, CptS 499, Special Problems, 2 credits, 5 students.
\item Fall 1999, CptS 317, Automata and Formal Languages, 3 credits, 
   12 students.
\item Spring 1999, CptS 455, Computer Networks, 3 credits,
   80 students.
\item Spring 1999, CptS 253, Java Programming Language, 3 credits, 
   50 students.
\item Fall 1998, EE 305, Introduction to Microprocessors, 2 credits,
   20 students.
\item Fall 1998, CptS 251, C Programming Language, 2 credits, 80 students.
\item Fall 1998, CptS 252, Introduction to Windows Development Programming,
  3 credits, 10 students.
\end{enumerate}

\vspace{0.25cm}

\noindent{\large\bf Software Disseminated}

\begin{enumerate}
\item Libraries for reading and viewing digital elevation maps.
\item 2D image and 3D volumetric compression software using
  DCT and fractal techniques.
\end{enumerate}

\vspace{0.25cm}

\noindent{\large\bf Committees}

\begin{itemize}
\item January 2000 -- present. Faculty Governance Committee.\\
  Provides a formal, recognized structure through which faculty can have
  direct input in administrative decision making.
\end{itemize}

\vspace{0.25cm}

\noindent{\large\bf Consulting}

\begin{itemize}
\item Smith-Root Inc,. 14014 NE Salmon Creek Avenue, Vancouver, WA 98686.\\
Helped design and port control software for an
electronic fish barrier. The system is now web based, and uses a 
client/server protocol for remote control.
\end{itemize}

%\vspace{0.5cm}
%\pagebreak

\noindent{\large\bf Reviewer}

\begin{itemize}
\item Information Processing Letters
\item ACM Transactions on Graphics
\item Graphics Interface
\item IEEE Visualization
\item Shape Modeling International
\item 2003 International Conference on Cyberworlds, 
   December 3-5, 2003, Singapore.
\end{itemize}

\vspace{0.25cm}

%\pagebreak

\noindent{\large\bf Awards}

\begin{itemize}
\item Curtis Fellowship, 1994, Washington State University.
\item Phi Beta Kappa, 1990, Alpha Chapter, University of Washington.
\item Deans List, 1990, University of Washington.
\item Golden Key National Honor Society, 1990, University of Washington.
\end{itemize}

%\vspace{0.25cm}

%\pagebreak

%\noindent{\large\bf Other Activities}\\
\noindent{\large\bf Professional Service}\\

\noindent{WSU Linux Users Group, co-founder:}
\begin{itemize}
\item Faculty Advisor (1998--1999)
\item President (1997--1998) 
\item Treasurer (1994--1997),
\end{itemize}

\noindent{1995 {\em SIGGRAPH} student volunteer.}\\

%\noindent{\large\bf Courses Able to Teach}
%
%\begin{itemize}
%\item Programming Languages (C, C++, Java, Common Lisp, Ada, Assembly, \ldots)
%\item Computer Graphics
%\item Computer Networks
%\item Distributed Computing
%\item Operating Systems
%\item Compiler Design
%\item Program Design, Data Structures
%\item Theory of Computation
%\item User Interface Programming (Windows, X Window System, \ldots)
%\item Numerical Computation, Numerical Analysis
%\item Mathematics (Calculus, Linear Algebra, \ldots)
%\end{itemize}


%\vspace{0.25cm}

%\noindent{\large\bf Computer Skills}

%\begin{itemize}
%\item Programming in the UNIX and the X Window System environment.
%\item UNIX and X Window System administration.
%%\item MS Windows programming.
%\item Distributed and threaded computing (PVM, Posix threads, \ldots).
%\item High and low level graphics programming (OpenGL, \ldots).
%\item Programming languages: C, C{\small++}, Java, Common Lisp, 
%       Assembly.
%\end{itemize}

\vspace{0.25cm}

\noindent{\large\bf References}\\

Charles R. Lang, Associate Professor,
School of Electrical Engineering and Computer Science,
Washington State University Vancouver,
14204 NE Salmon Creek Avenue,
Vancouver, Washington 98686-9600,
(360) 546-9632.\\

John C. Hart, Associate Professor,
Dept. of Computer Science,
University of Illinois,
3212 Digital Computer Lab.
1304 Springfield Avenue,
Urbana, Illinois 61801
(217) 333-8740.\\

Patrick J. Flynn, Associate Professor,
Department of Computer Science and Engineering,
University of Notre Dame,
384 Fitzpatrick Hall of Engineering,
Notre Dame, Indiana 46556,
(219) 631-8803.\\

%Phil Smith, President,
%Raster Graphics Inc,
%1804 SE First Street,
%Redmond, OR 97756,
%(541) 923-5530.

\end{document}

