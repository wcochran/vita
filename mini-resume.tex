\documentclass[10pt, oneside]{article}%{amsart}
\usepackage{graphicx}
\usepackage{amssymb}
\usepackage{fullpage}

%SetFonts

%SetFonts


\title{Dr. Wayne O. Cochran}
\author{Computer Scientist\\
       %Software Engineer\\
       Intel Sports, Intel Corporation\\
       %(360) 334-0200 (cell) \\
       {\tt wayne.cochran@intel.com}}
\date{\vspace{-5ex}}

\begin{document}
\thispagestyle{empty}
\maketitle
\thispagestyle{empty}

\section*{Interests and Skills}

\begin{description}
\item[\bf Computer Graphics] : research and development, 
   fast rasterization and low level optimization, 
   3D pipeline design,  GPU shaders, 
   mathematical modeling, volumetric rendering, intricate fractal models.
\item[\bf General Purpose GPU Programming] : %Algorithm development, 
    image processing, 
    physics simulation, video stitching, CUDA.
\item[\bf Image Processing / Computer Vision] : geometric transformations, color processing, 
   camera models, image reconstruction, compositing, 
   %frequency encoding, image reconstruction, compression, % \ldots
%\item[\bf Low-Level Vision] : f
   feature detection, projective geometry, dense stereo matching, 
   belief propagation for solving Markov Random Fields, stereo calibration, OpenCV.
\item[\bf Numerical Computing] : Optimization techniques, %Bayesian Networks, 
   parallel algorithms, numerical analysis.
\end{description}

\section*{Professional Experience}
\begin{description}
\item[\bf Software Engineer] 2017 - present,
Intel Sports, Intel Corporation. 
Research, development and implementation of live video processing
pipeline that captures, transforms, projects,
rectifies, stitches, encodes, and transmits
large stereo panorama streams.
\item[\bf Clinical Associate Professor] 1999-2017, 
Washington State University Vancouver.
Taught numerous courses at WSU that cover a wide
range of topics from the sophomore to graduate level that includes
Computer Graphics, Numerical Computing, Compiler Design, Theory of Computation.
%Operating Systems, and Computer Networks. 
%Research in intricate surface and texture
%modeling using recurrent fractal models. 
\item[\bf Software Engineer] 1990-1992,
Raster Graphics Inc. Design and implementation of rasterization firmware.
% for
%industrial graphics boards.
\end{description}

\section*{Education}
\begin{description}
\item[\bf Ph.D. Computer Science] 1998, Washington State University,
School of Electrical Engineering and Computer Science.
Dissertation Title: ``A Recurrent Modeling Toolset.''
\item[\bf M.S. Computer Science] 1994, Washington State University,
School of Electrical Engineering and Computer Science. Curtis Fellowship.
Thesis title: ``Fractal Volume Compression.''
\item[\bf B.S. Mathematics] cum laude, 1990.
University of Washington. 
Golden Key,
%National Honor Society, 
Dean's List, Phi Beta Kappa.
\end{description}

\enlargethispage{\baselineskip}

\section*{Selected Publications}
\begin{enumerate}
\item Matthew J. Lambert, Wayne O. Cochran. Kyle G. Olsen, Cynthia D. Cooper,
  Evidence for widespread subfunctionalization of splice forms in vertebrate genomes,
  {\em Genome Research.} 2015 May; 25(5): 624�632.
\item
Wayne O. Cochran, John C. Hart, Patrick J. Flynn,
Fractal Volume Compression,
{\em IEEE Transactions on Visualization and Computer Graphics}
December 1996.
\end{enumerate}

\end{document}  