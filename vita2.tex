%\documentstyle[12pt,vita,fullpage]{article}
\documentclass[10pt]{article}

\usepackage{vita}
%\usepackage{myfullpage}
\usepackage{fullpage}
\usepackage{graphicx}
\usepackage{textcomp}

\setlength{\parindent}{0em}

\title{Curriculum Vit\ae}
\author{Wayne O. Cochran\\
       Software Engineer\\
       Intel Sports, Intel Corporation\\
%      Intel\textsuperscript{\textregistered} Corporation\\
%       2111 N.E. 25th Avenue\\
% Hillsboro, OR 97214-5961, USA\\
       (360) 334-0200 (Cell) \\
       {\tt wayne.cochran@intel.com}}
%       {\tt https://ezekiel.encs.vancouver.wsu.edu/\~{}wayne}}
%      {\tt wayne.cochran@gmail.com}}

\begin{document}

\maketitle

%\noindent{\large\bf Current Appointment}\\

%Clinical Assistant Professor in the the School of
%Engineering and 
%Computer Science at Washington State University Vancouver. 
%%Appointment began in August, 1999.

%\vspace{0.5cm}

%%\noindent{\large\bf Research and Technical Interests}\\
\section*{Research and Technical Interests}

{\em Computer Graphics, Rasterization, 
GPU programming, Geometric Modeling, Image Processing, Mobile Application Development.}
My career in computer graphics started by 
implementing and optimizing rasterization firmware
for industrial graphics boards based on
TI's TMS34010 line of graphics processors.
Later I developed a 3D rendering system
that used the TMS34082 floating  point coprocessor.
Essentially I was programming GPU's long before
the current wave of GPU coding become popular.
Later, during graduate school, I developed an interest
in a class of fractal models that were used for compressing
images at unusually high bit rates. I extended these
algorithms (along with methods based on the DCT)
to compress 3-D volumetric data produced by CT
and MRI machines. Following this, I developed a variety of methods
based on fractal interpolation functions to model
and render intricate curves and surfaces.
%I remain interested in modeling 
%complex surfaces and textures with fractals and I am
%investigating techniques inspired by
%Wang Tiles to generate infinitely detailed structures.
Due to the availability of cheap video hardware,
GPU programming
has now extended beyond the realm of computer graphics.
General Purpose GPU (GPGPU) techniques are also
an interest of mine.
I designed and implemented the real-time video stitching algorithm
used by Intel Sports TrueVR 
%using nVidia's CUDA parallel computing platform for a 
stereo panorama camera system. I also helped develop and implement
the color processing pipeline for this system
which is used for broadcasting live sporting events as $180^\circ$ stereo
panoramic video for head mounted displays.
%I also enjoy programming mobile devices
%which utilize embedded GPU's.


\section*{Professional Experience}

\begin{itemize}
\item {\bf Software Engineer}, 2017 - 2020,
Intel Sports, Intel Corporation. Design
and implement video processing software for 
stereo panorama system. Create
GPU pipeline modules for a live video processing
pipeline that captures, transforms, projects,
rectifies, stitches, encodes, and transmits
large stereo panorama streams at 60 FPS that is
targeted for viewing
with a head mounted display. 
Construction of computer vision pipeline for detecting
athletes from multi-camera video and reconstructing 3D skeletons.
Expertise
includes computer graphics, image processing,
low-level computer vision, and GPU programming.


\item {\bf Clinical Associate Professor}, 1999-2017,
Washington State University Vancouver.
%My original position was on a tenure track, but was later switched
%to a clinical position to match teaching interests. 
I taught 17 different computer science courses at WSU that cover a wide
range of topics from the sophomore to graduate level that includes
Computer Graphics, Numerical Computing, Compiler Design, Theory of Computation,
Operating Systems, and Computer Networks.
Promoted
to the Associate Professor rank in~2015.
I was heavily involved in program development for 
the computer science program which started in the fall of 1999.
Any research performed as clinical professor involves intricate surface and texture
modeling using recurrent fractal models. % and Wang Tiles.
Maintained a primary interest in GPU programming for both
graphical applications and general purpose parallel computing.

\item {\bf Instructor}, 1998-1999,
Washington State University. I performed research and taught five
courses including Computer Networks, Introduction to Microprocessors,
and several introductory programming classes.

%courses in Introduction to Microprocessors, C Programming,
%Windows Programming, Computer Networks, and Java Programming.

\item {\bf Research Assistant}, 1992-1998,
Washington State University.
My research included volumetric compression using fractals,
and other techniques using fractals to model complex geometry
({\it e.g.,} rough curves and surfaces). Research was part of the
{\em Recurrent Modeling Project} funded by Intel and a grant
from the NSF.
%Also acted as substitute teacher for Computer Graphics, 
%Advanced Computer Graphics, and Numerical Computing. 
%\item {\bf Research/Teaching Assistant}, 1992--1994, 
%Washington State University.
Contributed research for and implementation of knowledge based systems.
%Lab instructor for
%introductory computer programming courses.

\item {\bf Software Engineer}, 1990-1992,
Raster Graphics Inc
(assets acquired by Peritek Corporation in 2001, now
located in Berkeley, CA)
{\tt http://www.rastergraf.com}.
%1804-P SE First St.
%Redmond, Oregon 97756
%% 1804 SE First Street
%%Redmond, OR 97756.
%(541) 923-5530
Job Overview: 
Design and implementation of rasterization firmware for
industrial graphics boards. Design and implementation of
3D rendering libraries.
\end{itemize}


\section*{Education}

\begin{itemize}
\item {\bf Ph.D.} Computer Science, 1998,
Washington State University. \\
%School of Electrical Engineering and Computer Science.
Dissertation Title: ``A Recurrent Modeling Toolset.''
%Advisor: John C. Hart.
%GPA: 3.98. 

\item {\bf M.S.} Computer Science, 1994,
Washington State University. Curtis Fellowship. \\
%School of Electrical Engineering and Computer Science.
Thesis title: ``Fractal Volume Compression.''

%Advisor: John C. Hart.
%GPA: 4.00.

\item {\bf B.S.} Mathematics, cum laude, 1990.
% College of Arts and Sciences,
University of Washington. 
Golden Key National Honor Society, Dean's List, Phi Beta Kappa.
%GPA: 3.75.
\end{itemize}

%\subsection*{Awards}
%
%\begin{itemize}
%\item Nominated for the Student Choice Award for Teaching
%  Excellence by the Associated Students of WSU Vancouver (ASWSUV).
%\item Curtis Fellowship, 1994, Washington State University.
%\item Phi Beta Kappa, 1990, Alpha Chapter, University of Washington.
%\item Deans List, 1990, University of Washington.
%\item Golden Key National Honor Society, 1990, University of Washington.
%\end{itemize}


\section*{Patents}

\begin{enumerate}
\item Intel TrueVR System
\item Apparatus and System for Hybrid Real-Time Playback and Progressive
  Download of Point Cloud Sequence Data
\end{enumerate}

%%\noindent{\large\bf Publications}
\section*{Publications}

\begin{enumerate}

\item Matthew J. Lambert, Wayne O. Cochran. Kyle G. Olsen, Cynthia D. Cooper,
  Evidence for widespread subfunctionalization of splice forms in vertebrate genomes,
  {\em Genome Research.} 2015 May; 25(5): 624D632.

\item Wayne O. Cochran,
Recurrent Interpolation Surfaces,
{\em Proceedings of the Western Computer Graphics Symposium,}
March 2003, pp.~9--15.

\item Wayne. O. Cochran, R. R. Lewis, J. C. Hart,
The Normal of a Fractal Surface,
{\em The Visual Computer,} vol. 17, no. 4, April 2001, pp.~209--218.

\item Wayne O. Cochran,
Fractal Interpolation Surfaces for Digital Elevation Maps,
{\em Proceedings of the Western Computer Graphics Symposium,}
March 2001, pp.~8--14.

\item Wayne O. Cochran,
A Recurrent Modeling Toolset, Ph.D. dissertation.
Washington State University, December 1998.

\item
Wayne O. Cochran, John C. Hart, Patrick J. Flynn,
On Approximating Rough Curves with Fractal Functions,
{\em Proceedings of Graphics Interface,}
June 1998.

\item
J.C. Hart, P.J. Flynn, W.O. Cochran. 
Similarity Hashing: A Model-Based Vision Solution to the Inverse 
Problem of Recurrent Iterated Function Systems. 
{\em Fractals 5} April 1997, pp.~39-50.

\item
Wayne O. Cochran, John C. Hart, Patrick J. Flynn,
Hashing Fractal Functions
{\em Proceedings of the Western Computer Graphics Symposium,}
April 1997, pp.~69--78.

\item
Wayne O. Cochran, John C. Hart, Patrick J. Flynn,
Fractal Volume Compression,
{\em IEEE Transactions on Visualization and Computer Graphics}
2 (4),
%vol. 2, no. 4, 
December 1996, pp.~313--322.

\item
Wayne O. Cochran, John C. Hart, Patrick J. Flynn,
Similarity and Affinity Hashing,
{\em Proceedings of the Western Computer Graphics Symposium,}
March 1996, pp.~89--100.

\item
Wayne O. Cochran, J.C. Hart and P.J. Flynn. Recurrent Modeling. 
{\em Intel Forum: Enabling Live Media in Cyberspace,} 
invited poster. January 1996.

\item
Wayne O. Cochran, John C. Hart, Patrick J. Flynn,
Principal Component Classification for Fractal Volume Compression,
{\em Proceedings of the Western Computer Graphics Symposium,}
March 1995, \mbox{pp.~9--18}.

\end{enumerate}


\section*{Consulting}

\begin{itemize}

%\item 3D-4U and Integrated Engineering Solutions (IES), 
%1610 NE Eastgate Blvd.
%Suite 440
%Pullman, WA 99163,

\item Voke VR (acquired by Intel 2016) 
3201 Scott Blvd, Santa Clara, CA, USA 95054
%\verb@www.3d-4u.com.@, \verb@www.ie-sol.com.@
Designed and implemented a video stitching algorithm for
a stereo panorama camera system. The solution was
implemented using NVidia's CUDA parallel computing platform
for Telsa GPU-based systems.


\item GeoMonkey, Inc, 
5512 NE 109th Ct. Ste 101
Vancouver, WA 98662,
(360) 718-8120,
{\tt \verb@www.geomonkey.com@.}  
  Converted large KML polygonal datasets into a form used for
  fast multiresolution viewing in Google Maps.
  Also helped implement the {\tt MapWithUs GIS} iPhone app using Apple's core location technology
  and Google's map API.
\item Smith-Root Inc, 14014 NE Salmon Creek Avenue, Vancouver, WA 98686.\\
Helped design and port control software for an
electronic fish barrier. The system is now web based, and uses a 
client/server protocol for remote control.
\end{itemize}

%\vspace{0.5cm}
%\pagebreak


%\noindent{\large\bf Reviewer}
%%\noindent{\large\bf Professional Service}
\section*{Professional Service}

\begin{itemize}
\item Paper Chair for Thirteen Annual {\em Consortium for Computing Sciences in Colleges} (CCSC)
Northwestern Regional Conference 2011. Responsible for collecting all submitted papers, procuring referees, and organizing the paper acceptance committee.

\item Paper Referee for the following Journals and Conferences:
\begin{itemize}
\item ACM SIGGRAPH
\item ACM Transactions on Graphics
\item IEEE Transactions on Visualization and Computer Graphics
\item IEEE Transactions on Pattern Analysis and Machine Intelligence
\item Information Processing Letters
\item Graphics Interface
\item IEEE Visualization
\item Shape Modeling International
\item International Conference on Cyberworlds
%   (2003, Singapore; 2004, Tokyo; 2005, Singapore).
%\item 2003 International Conference on Cyberworlds, 
%   December 3-5, 2003, Singapore.
%\item 2004 International Conference on Cyberworlds, 
%   November 10-20, 2004, Tokyo.
\item Perspectives on Science and Christian Faith, The Journal of the
  American Scientific Affiliation.
\end{itemize}
\end{itemize}


%\noindent{\large\bf Other Activities}\\
%\noindent{\large\bf Professional Service}\\

%\noindent{WSU Linux Users Group, co-founder:}
%\begin{itemize}
%\item Faculty Advisor (1998--1999)
%\item President (1997--1998) 
%\item Treasurer (1994--1997),
%\end{itemize}

%\noindent{1995 {\em SIGGRAPH} student volunteer.}\\

%\noindent{\large\bf Courses Able to Teach}
%
%\begin{itemize}
%\item Programming Languages (C, C++, Java, Common Lisp, Ada, Assembly, \ldots)
%\item Computer Graphics
%\item Computer Networks
%\item Distributed Computing
%\item Operating Systems
%\item Compiler Design
%\item Program Design, Data Structures
%\item Theory of Computation
%\item User Interface Programming (Windows, X Window System, \ldots)
%\item Numerical Computation, Numerical Analysis
%\item Mathematics (Calculus, Linear Algebra, \ldots)
%\end{itemize}


%%\vspace{0.25cm}
%%\noindent{\large\bf References}\\

\section*{References}

Upon request.

%Charles R. Lang\\ 
%Associate Professor \\
%School of Engineering and Computer Science \\
%Washington State University Vancouver \\
%14204 NE Salmon Creek Avenue \\
%Vancouver, Washington 98686-9600 \\
%\verb$dick_lang@vancouver.wsu.edu$ \\
%(360) 546-9632\\
%
%\pagebreak
%
%Orest Pilskalns\\
%Chief Executive Officer\\
%mapwith.us\\
%5512 NE 109th Court Suite J\\
%Vancouver, WA 98662 \\
%{\tt orest@mapwith.us} \\
%(360) 901-8813\\
%
%Sankar Jayaram\\  
%Co-Founder, President\\ % (Professor, MME WSU)\\
%%{\tt sjayaram@ie-sol.com} \\
%Uma Jayaram \\
%Co-Founder, EVP \\ %(Associate Professor, MME WSU)\\
%%{\tt ujayaram@ie-sol.com} \\
%{\tt uma.jayaram@intel.com} \\
%Voke VR / Intel \\
%3201 Scott Blvd, \\
%Santa Clara, CA, \\
%USA 95054 \\
%
%%3D-4U and Integrated Engineering Solutions (IES) \\
%%1610 NE Eastgate Blvd. \\
%%Suite 440 Pullman, WA 99163\\
%%(509) 335-6454\\
%
%%Roger C. Ray\\
%%#Principal Engineer, Intel (retired)\\
%%{\tt roger@anastasia.com} \\
%%(503) 292-1476.\\
% 
%John C. Hart\\
%Professor, Associate Dean of the Graduate College\\
%Department of Computer Science\\
%University of Illinois\\
%%3227 Siebel Center \\
%201 N. Goodwin \\
%Urbana, IL 61801 \\
%%{\tt jch@cs.uiuc.edu} \\
%{\tt jch@illinois.edu} \\
%(217) 333-8740 \\
%
%%Patrick J. Flynn, Associate Professor,
%%Department of Computer Science and Engineering,
%%University of Notre Dame,
%%384 Fitzpatrick Hall of Engineering,
%%Notre Dame, Indiana 46556,
%%(219) 631-8803.\\
%
%%Phil Smith, President,
%%Raster Graphics Inc,
%%1804 SE First Street,
%%Redmond, OR 97756,
%%(541) 923-5530.
%% now http://www.rastergraf.com/index.htm (7/8/2014)

\end{document}

