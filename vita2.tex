%\documentstyle[12pt,vita,fullpage]{article}
\documentclass[10pt]{article}

\usepackage{vita}
%\usepackage{myfullpage}
\usepackage{fullpage}
\usepackage{graphicx}
\usepackage{textcomp}

\setlength{\parindent}{0em}

\title{Curriculum Vit\ae}
\author{Wayne O. Cochran\\
       Software Engineer\\
       Intel Sports, Intel Corporation\\
%      Intel\textsuperscript{\textregistered} Corporation\\
%       2111 N.E. 25th Avenue\\
% Hillsboro, OR 97214-5961, USA\\
       (360) 334-0200 (Cell) \\
       {\tt wayne.cochran@intel.com}}
%       {\tt https://ezekiel.encs.vancouver.wsu.edu/\~{}wayne}}
%      {\tt wayne.cochran@gmail.com}}

\begin{document}

\maketitle

%\noindent{\large\bf Current Appointment}\\

%Clinical Assistant Professor in the the School of
%Engineering and 
%Computer Science at Washington State University Vancouver. 
%%Appointment began in August, 1999.

%\vspace{0.5cm}

%%\noindent{\large\bf Research and Technical Interests}\\
\section*{Research and Technical Interests}

{\em Computer Graphics, Rasterization, 
GPU programming, Geometric Modeling, Image Processing, Mobile Application Development.}
My career in computer graphics started by 
implementing and optimizing rasterization firmware
for industrial graphics boards based on
TI's TMS34010 line of graphics processors.
Later I developed a 3D rendering system
that used the TMS34082 floating  point coprocessor.
Essentially I was programming GPU's long before
the current wave of GPU coding become popular.
Later, during graduate school, I developed an interest
in a class of fractal models that were used for compressing
images at unusually high bit rates. I extended these
algorithms (along with methods based on the DCT)
to compress 3-D volumetric data produced by CT
and MRI machines. Following this, I developed a variety of methods
based on fractal interpolation functions to model
and render intricate curves and surfaces.
%I remain interested in modeling 
%complex surfaces and textures with fractals and I am
%investigating techniques inspired by
%Wang Tiles to generate infinitely detailed structures.
Due to the availability of cheap video hardware,
GPU programming
has now extended beyond the realm of computer graphics.
General Purpose GPU (GPGPU) techniques are also
an interest of mine.
I designed and implemented the real-time video stitching algorithm
used by Intel Sports TrueVR 
%using nVidia's CUDA parallel computing platform for a 
stereo panorama camera system. I also helped develop and implement
the color processing pipeline for this system
which is used for broadcasting live sporting events as $180^\circ$ stereo
panoramic video for head mounted displays.
%I also enjoy programming mobile devices
%which utilize embedded GPU's.

%%%\vspace{0.5cm}
%%%\noindent{\large\bf Teaching Interest and Goals}\\
%\section*{Teaching Interest and Goals}
%
%I have taught 17 different computer science courses at WSU from 1998 to the present that cover a wide
%range of topics from the sophomore to graduate level. The topics include computability and complexity
%theory and algorithm analysis, scientific and numerical computing,  programming language design
%and translation, application programming, computer graphics, and mobile application design.
%Many of these subjects involve rapidly changing technology which are both exciting and challenging
%to keep pace with. For example the constant evolution of graphics hardware and API's and the
%variety of GPU platforms requires yearly redesigns of several of my courses. The current 
%explosion of web and mobile frameworks, database and cloud storage techniques, and embedded
%devices requires continual learning. It is important that I provide students with current 
%state-of-the-art material and that I create interesting and relevant programming projects to prepare
%them for their careers. At the same time, its crucial that I do not focus on specific technologies
%due to their transient nature, but provide the student with a foundation that will allow
%them to quickly absorb and even create new technology. 
%
%Predominantly I enjoy teaching upper division courses with students that have already began
%to form a foundation in computer science and software design. With a background and love
%for mathematics I enjoy the epiphanies that students experience when they see the power
%and usefulness of the math and theoretical knowledge they have been previously armed with.
%Every year I have students who flourish and genuinely enjoy computer science and it is
%these students who make what I do worthwhile.
%  
%%%\vspace{0.5cm}
%%%\noindent{\large\bf Courses WSU Vancouver EECS/ENCS 1999-2014}
%\subsection*{Courses Taught at WSU Vancouver EECS/ENCS 1999-2014}
%
%\begin{description}
%  \item[CS 223 Advanced Data Structures] (2007, 2012-2016) Abstract data types, balanced binary trees (splay
%     trees, AVL trees, red-black trees), hash-tables, heaps, graph traversal algorithms (Dijkstra's algorithm) 
%     (2007 C++, 2012-2016 Java).
%  \item[CS 330 Numerical Computing] (2008-2016) Power and limitation of numerical solutions; design, analysis 
%      and implementation of numerical algorithms.
%  \item[CS 452/542 Computer Graphics] (2001-2016) OpenGL %\textregistered 
%      graphics pipeline, geometric modeling and
%      transformations, visibility, shading, texturing (conjoint grad course).
%  \item[CS 224 Programming Tools] (2008-2015, 2017) Unix file system, shell scripting, debugging, version control,
%     graphical user interface programming, model-view-controller (2008-2009 wxWidgets, 2010-2014 Cocoa/Xcode).
%  \item[CS 483,458 Mobile Application Development] (2012-2017) Model-view-controller (MVC), event driven
%     programming, mobile application user interfaces and frameworks, data persistence, 
%     front-end and back-end design, RESTful network API's, sensors (iOS, Xcode, Cocoa Touch).
%  \item[CS 421 Software Engineering in Practice] (2015-2076) Senior Capstone course, implementation of
%     industry sponsored senior projects, software management tools, version control.
%  \item[CS 452 Compiler Design] (2007, 2009, 2011, 2013, 2014) Lexical and syntax analysis, parsing tools
%     (lex/yacc/bison), abstract syntax trees, symbol tables, intermediate representation (IR) code generation,
%     static single assignment (LLVM), register allocation.
%  \item[CS 317 Automata and Formal Languages] (1999-2006, 2008, 2010) Formal languages and 
%     models of computation; finite automata, 
%     pushdown automata, Turing machines, regular languages, context free languages, parsing algorithms, 
%     computability (Halting Problem).
%  \item[CS 355 Programming Language Design] (2001-2011,2015-2016) 
%     Language syntax, semantics, parsing strategies, translation
%     and interpretation, memory layout and management, statements, expressions, control-flow, support
%     for object-oriented programming, functional languages, logic programming.
%  \item[CS 516 Theory of Computation] (2006, 2009) Computability and complexity theory, decidability, 
%      Church-Turing Thesis, space and time complexity hierarchies (P, NP, co-NP, P-space), NP and P-space
%      completeness.
%  \item[CS 548 Advanced Computer Graphics] (2003, 2005, 2006, 2008, 2010, 2012) Ray tracing, procedural 
%      texturing (Perlin noise), advanced mesh data structures, subdivision surfaces, GPU programming.
%  \item[CS 251 C Programming for Engineers] (2007) Computation and memory model,
%    syntax, expressions, control flow, addressing and pointers, modularization, simple data structures, 
%    numerical methods for scientific and engineering applications.
%  \item[CptS 460 Operating Systems and Computer Architecture] (2000, 2001) Process management, 
%     preemptive multitasking, scheduling, concurrency issues, memory management, virtual memory, file system,
%     inout/output, device drivers (Minix kernel).
%  \item[CptS 499 Professional Practice] (2000, 2001) Programming methods and practical exercises
%     for ``real world'' problems using
%     Java (2000) and C++ (2001) (co-taught with adjunct Roger Ray).
%  \item[Math 210 Introduction to Mathematics] (2007) General Education course on mathematics for non-technical
%     students.
%\end{description}
%
%%%\noindent{\large\bf Courses WSU Pullman EECS / 1998-1999}
%\subsection*{Courses Taught at WSU Pullman EECS from 1998-1999}
%
%\begin{description}
%  \item[CptS 252 Windows Programming] (1998) Win32 API for C. Windows environment, event-driven programming,
%     event callbacks, user interface programming.
%  \item[CptS 251 C Programming] (1998) C programming for engineers. Computation and memory model,
%    syntax, expressions, control flow, modularization, simple data structures, numerical methods.
%  \item[EE 305 Introduction to Microprocessors] (1998) Structured computer organization, simple digital
%     circuits, memory, microprocessors (Z80), assembly language, operating systems, applications.
%  \item[CptS 253 Java Programming] (1999) Java programming language, object oriented programming,
%     java applications and applets, abstract window toolkit (AWT).
%  \item[CptS 455 Computer Networks] (1999) OSI network layer model, TCP/IP, physical layer encoding, 
%      aloha, ethernet, network layer (IP, ICMP), transport layer (TCP, UDP), connection vs connectionless,
%      streams vs datagrams, presentation layer, application layer (NFS, FTP, HTTP), Socket programming.
%\end{description}
%
%%%\noindent{\large\bf Graduate Student Advising}
%\subsection*{Graduate Student Advising (Chair)}
%\begin{itemize}
%\item Jason Neufeld (on hiatus, currently working at Google) MS Topic: {\em Tiling Techniques for Texture Maps.}
%\item Michael Persons, MS May 2010.
%  {\em Texture Synthesis with Wang Tiles and Recurrent Iterated Function Systems.}
%\item Gunay Uyan, MS December 2005. {\em Efficient Wang-Tiling and Real Time
%   Rendering of Lambertian Reflectance Maps.}
%\item Ryan Tindall, MS December 2005. {\em Graphics Hardware Acceleration
%   of the Finite Difference Time Domain (FDTD) Algorithm.}
%\end{itemize}
%
%\subsection*{Graduate Thesis Committee}
%  \begin{itemize}
%  \item Kyle Siehl, MS December 2016, Advisor: Xinghui Zhao, 
%   {\em Archon -- A Framework for Dynamically-Tuned CPU-GPU Hybridization.}
%  \item Ellen Porter, PhD August 2016, Advisor: Robert Lewis,
%   {\em Communication Avoiding Ray Tracing for Exascale Computing}.
%  \item Rick Riensche, PhD May 2013, Advisor: Bob Lewis. {\em Modeling and Rendering of
%      Fibrous Materials.}
%%  \item Evan Dickenson / To complete 2013, Advisor: Scott Wallace. Machine Learning.
%  \item Travis Hall, MS May 2012, Advisor: David Chiu. {\em A Cost-Driven 
%      Replacement Policy for a Hierarchical Key-Value Store.}
%  \item Farhana Kabir, MS July 2012, Advisor: David Chiu. {\em A self-managed cloud cache 
%      for accelerating data intensive applications.}
%  \item Joseph Sturtevant, May 2011. Advisor: Orest Pilskalns. {\em Auto-Generating 
%      a Multi-Tiered Data Acquisition System.}
%  \item Michael Heilmann, May 2011, Advisor: Orest Pilskalns. {\em Architectural analysis of 
%      generated UML documentation representing physical components.}
%  \item Justin Morgan, May 2010. Advisor: George Fletcher, {\em Visual Language for 
%      Exploring Massive RFD Data Sets.}
%  \item James Van Boxtel, May 2010. Advisor: Scott Wallace. {\em An Evaluation of 
%      Interactive Curriculum Using the Java Instructional Game Engine.}
%  \item Paul Anthony Mancill Jr., December 2010. Advisor:Scott Wallace. {\em An Exploration 
%     of Naive Bayesian Classification Augmented with Confidence Intervals.}
%  \item Hadresh Patel, December 2010. Advisor: Scott Wallace. {\em Machine Learning Approach 
%      to Barcode Detection and Stamp Identification.} 
%  \item Xiaogang Yang, August 2010. Advisor: Wen Zhan Song. {\em Liveweb: A Sensorweb Portal 
%      Sensing the World in Real Time.}
% %\item Kevin Karpenske, August 2011. Advisor: Orest Pilskalns, Topics in Software Engineering
%  \item Adam McDonald, Map 2010. Advisor: Orest Pilskalns. {\em An Integrated UML Based 
%      Model for Design Analysis.}
%  \item Gang Lu, December 2010. Advisor: Wen Zhan Song. {\em Basic Components Design For Smart Grid.}
% % \item Daniel Best, May 2008 / Summer 2009. Advisor: Bob Lewis. Topic: Scientific Visualization.
%  \item Benjamin Eitzen,  August 2007, Advisor: Bob Lewis. {\em GPUPY: Efficiently Using a 
%    GPU with Python.}
%  \item Daniel Williams, August 2007,  Advisor: Orest Pilskalns. {\em MS Design 
%    Analysis Techniques for Software Quality Enhancement.}
%  \item Fengua Yuan, August 2007. Advisor: Wen Zhan Song. {\em Lightweight Network Management 
%     Design for Wireless Sensor Network.}
%\item James Edwards, MS December 2005. {\em A Hardware
%    Implementation of a Multilevel B-Spline Shader.}
%\item Randolf Schwartz, MS May 2005. {\em MCNPVIZ: A Program for the
%    Interactive Display of Monte Carlo N-Particle Geometry.}
%\item Shuangshuang Jin, MS 2003, {\em A Comparison of
%    Algorithms for Vertex Normal Computation.}
%\item Masaki Kameya, PhD 2002, {\em A Smooth,
%   Efficient Representation of Reflectance.}
%\item Cheng-Chih Fan-Chiang. MS 2002, {\em Star-Flower
%   Subdivision.}
%\item Frank Taylor, MS 2001, {\em Enhancements to a 
%   Virtual Assembly Environment for Simulation of Heavy
%   Machinery Assembly.}   
%\item Nathan Carr, MS 2000, {\em Procedural Solid
%   Texture Mapping Using Existing Computer Graphics API's.}
%\item Shuyang Li, MS 2000, {\em Computing Reflectance
%   from Height Fields.}
%\end{itemize}
%
%\subsection*{Committees}
%
%\begin{itemize}
%\item Computer Science Curriculum Committee (annually).
%\item WSUV Scholarship Committee (2006 -- 2017).
%\item WSUV Research Showcase judge (2010 -- 2012)
%\item Consortium for Computing Sciences in Colleges (CCSC) NW Region,
%   papers chair for 12th Annual Conference (2011).
%\item Chair of Computer Science Graduate Studies Committee (2005 -- 2010).
%\item General Education Assessment Committee for WSUV (2006, 2007).
%\item WSUV IT Director Search (2007).
%\item WSUV Lower Division Planning Committee member (2005).
%\end{itemize}

\section*{Professional Experience}

\begin{itemize}
\item {\bf Software Engineer}, 2017 - present,
Intel Sports, Intel Corporation. Design
and implement video processing software for 
stereo panorama system. Predominantly create
GPU pipeline modules for a live video processing
pipeline that captures, transforms, projects,
rectifies, stitches, encodes, and transmits
large stereo panorama streams at 60 FPS that is
targeted for viewing
with a head mounted display. Expertise
includes computer graphics, image processing,
low-level computer vision, and GPU programming.


\item {\bf Clinical Associate Professor}, 1999-2017,
Washington State University Vancouver.
%My original position was on a tenure track, but was later switched
%to a clinical position to match teaching interests. 
I taught 17 different computer science courses at WSU that cover a wide
range of topics from the sophomore to graduate level that includes
Computer Graphics, Numerical Computing, Compiler Design, Theory of Computation,
Operating Systems, and Computer Networks.
Promoted
to the Associate Professor rank in~2015.
I was heavily involved in program development for 
the computer science program which started in the fall of 1999.
Any research performed as clinical professor involves intricate surface and texture
modeling using recurrent fractal models. % and Wang Tiles.
Maintained a primary interest in GPU programming for both
graphical applications and general purpose parallel computing.

\item {\bf Instructor}, 1998-1999,
Washington State University. I performed research and taught five
courses including Computer Networks, Introduction to Microprocessors,
and several introductory programming classes.

%courses in Introduction to Microprocessors, C Programming,
%Windows Programming, Computer Networks, and Java Programming.

\item {\bf Research Assistant}, 1992-1998,
Washington State University.
My research included volumetric compression using fractals,
and other techniques using fractals to model complex geometry
({\it e.g.,} rough curves and surfaces). Research was part of the
{\em Recurrent Modeling Project} funded by Intel and a grant
from the NSF.
%Also acted as substitute teacher for Computer Graphics, 
%Advanced Computer Graphics, and Numerical Computing. 
%\item {\bf Research/Teaching Assistant}, 1992--1994, 
%Washington State University.
Contributed research for and implementation of knowledge based systems.
%Lab instructor for
%introductory computer programming courses.

\item {\bf Software Engineer}, 1990-1992,
Raster Graphics Inc
(assets acquired by Peritek Corporation in 2001, now
located in Berkeley, CA)
{\tt http://www.rastergraf.com}.
%1804-P SE First St.
%Redmond, Oregon 97756
%% 1804 SE First Street
%%Redmond, OR 97756.
%(541) 923-5530
Job Overview: 
Design and implementation of rasterization firmware for
industrial graphics boards. Design and implementation of
3D rendering libraries.
\end{itemize}


\section*{Education}

\begin{itemize}
\item {\bf Ph.D.} Computer Science, 1998,
Washington State University. \\
%School of Electrical Engineering and Computer Science.
Dissertation Title: ``A Recurrent Modeling Toolset.''
%Advisor: John C. Hart.
%GPA: 3.98. 

\item {\bf M.S.} Computer Science, 1994,
Washington State University. Curtis Fellowship. \\
%School of Electrical Engineering and Computer Science.
Thesis title: ``Fractal Volume Compression.''

%Advisor: John C. Hart.
%GPA: 4.00.

\item {\bf B.S.} Mathematics, cum laude, 1990.
% College of Arts and Sciences,
University of Washington. 
Golden Key National Honor Society, Dean's List, Phi Beta Kappa.
%GPA: 3.75.
\end{itemize}

%\subsection*{Awards}
%
%\begin{itemize}
%\item Nominated for the Student Choice Award for Teaching
%  Excellence by the Associated Students of WSU Vancouver (ASWSUV).
%\item Curtis Fellowship, 1994, Washington State University.
%\item Phi Beta Kappa, 1990, Alpha Chapter, University of Washington.
%\item Deans List, 1990, University of Washington.
%\item Golden Key National Honor Society, 1990, University of Washington.
%\end{itemize}


%%\noindent{\large\bf Professional Experience}

%%\noindent{\large\bf Publications}
\section*{Publications}

\begin{enumerate}

\item Matthew J. Lambert, Wayne O. Cochran. Kyle G. Olsen, Cynthia D. Cooper,
  Evidence for widespread subfunctionalization of splice forms in vertebrate genomes,
  {\em Genome Research.} 2015 May; 25(5): 624�632.

\item Wayne O. Cochran,
Recurrent Interpolation Surfaces,
{\em Proceedings of the Western Computer Graphics Symposium,}
March 2003, pp.~9--15.

\item Wayne. O. Cochran, R. R. Lewis, J. C. Hart,
The Normal of a Fractal Surface,
{\em The Visual Computer,} vol. 17, no. 4, April 2001, pp.~209--218.

\item Wayne O. Cochran,
Fractal Interpolation Surfaces for Digital Elevation Maps,
{\em Proceedings of the Western Computer Graphics Symposium,}
March 2001, pp.~8--14.

\item Wayne O. Cochran,
A Recurrent Modeling Toolset, Ph.D. dissertation.
Washington State University, December 1998.

\item
Wayne O. Cochran, John C. Hart, Patrick J. Flynn,
On Approximating Rough Curves with Fractal Functions,
{\em Proceedings of Graphics Interface,}
June 1998.

\item
J.C. Hart, P.J. Flynn, W.O. Cochran. 
Similarity Hashing: A Model-Based Vision Solution to the Inverse 
Problem of Recurrent Iterated Function Systems. 
{\em Fractals 5} April 1997, pp.~39-50.

\item
Wayne O. Cochran, John C. Hart, Patrick J. Flynn,
Hashing Fractal Functions
{\em Proceedings of the Western Computer Graphics Symposium,}
April 1997, pp.~69--78.

\item
Wayne O. Cochran, John C. Hart, Patrick J. Flynn,
Fractal Volume Compression,
{\em IEEE Transactions on Visualization and Computer Graphics}
2 (4),
%vol. 2, no. 4, 
December 1996, pp.~313--322.

\item
Wayne O. Cochran, John C. Hart, Patrick J. Flynn,
Similarity and Affinity Hashing,
{\em Proceedings of the Western Computer Graphics Symposium,}
March 1996, pp.~89--100.

\item
Wayne O. Cochran, J.C. Hart and P.J. Flynn. Recurrent Modeling. 
{\em Intel Forum: Enabling Live Media in Cyberspace,} 
invited poster. January 1996.

\item
Wayne O. Cochran, John C. Hart, Patrick J. Flynn,
Principal Component Classification for Fractal Volume Compression,
{\em Proceedings of the Western Computer Graphics Symposium,}
March 1995, \mbox{pp.~9--18}.

\end{enumerate}


\section*{Consulting}

\begin{itemize}

%\item 3D-4U and Integrated Engineering Solutions (IES), 
%1610 NE Eastgate Blvd.
%Suite 440
%Pullman, WA 99163,

\item Voke VR (acquired by Intel 2016) 
3201 Scott Blvd, Santa Clara, CA, USA 95054
%\verb@www.3d-4u.com.@, \verb@www.ie-sol.com.@
Designed and implemented a video stitching algorithm for
a stereo panorama camera system. The solution was
implemented using NVidia's CUDA parallel computing platform
for Telsa GPU-based systems.


\item GeoMonkey, Inc, 
5512 NE 109th Ct. Ste 101
Vancouver, WA 98662,
(360) 718-8120,
{\tt \verb@www.geomonkey.com@.}  
  Converted large KML polygonal datasets into a form used for
  fast multiresolution viewing in Google Maps.
  Also helped implement the {\tt MapWithUs GIS} iPhone app using Apple's core location technology
  and Google's map API.
\item Smith-Root Inc, 14014 NE Salmon Creek Avenue, Vancouver, WA 98686.\\
Helped design and port control software for an
electronic fish barrier. The system is now web based, and uses a 
client/server protocol for remote control.
\end{itemize}

%\vspace{0.5cm}
%\pagebreak


%\noindent{\large\bf Reviewer}
%%\noindent{\large\bf Professional Service}
\section*{Professional Service}

\begin{itemize}
\item Paper Chair for Thirteen Annual {\em Consortium for Computing Sciences in Colleges} (CCSC)
Northwestern Regional Conference 2011. Responsible for collecting all submitted papers, procuring referees, and organizing the paper acceptance committee.

\item Paper Referee for the following Journals and Conferences:
\begin{itemize}
\item ACM SIGGRAPH
\item ACM Transactions on Graphics
\item IEEE Transactions on Visualization and Computer Graphics
\item IEEE Transactions on Pattern Analysis and Machine Intelligence
\item Information Processing Letters
\item Graphics Interface
\item IEEE Visualization
\item Shape Modeling International
\item International Conference on Cyberworlds
%   (2003, Singapore; 2004, Tokyo; 2005, Singapore).
%\item 2003 International Conference on Cyberworlds, 
%   December 3-5, 2003, Singapore.
%\item 2004 International Conference on Cyberworlds, 
%   November 10-20, 2004, Tokyo.
\item Perspectives on Science and Christian Faith, The Journal of the
  American Scientific Affiliation.
\end{itemize}
\end{itemize}


%\noindent{\large\bf Other Activities}\\
%\noindent{\large\bf Professional Service}\\

%\noindent{WSU Linux Users Group, co-founder:}
%\begin{itemize}
%\item Faculty Advisor (1998--1999)
%\item President (1997--1998) 
%\item Treasurer (1994--1997),
%\end{itemize}

%\noindent{1995 {\em SIGGRAPH} student volunteer.}\\

%\noindent{\large\bf Courses Able to Teach}
%
%\begin{itemize}
%\item Programming Languages (C, C++, Java, Common Lisp, Ada, Assembly, \ldots)
%\item Computer Graphics
%\item Computer Networks
%\item Distributed Computing
%\item Operating Systems
%\item Compiler Design
%\item Program Design, Data Structures
%\item Theory of Computation
%\item User Interface Programming (Windows, X Window System, \ldots)
%\item Numerical Computation, Numerical Analysis
%\item Mathematics (Calculus, Linear Algebra, \ldots)
%\end{itemize}


%%\vspace{0.25cm}
%%\noindent{\large\bf References}\\

\section*{References}

Upon request.

%Charles R. Lang\\ 
%Associate Professor \\
%School of Engineering and Computer Science \\
%Washington State University Vancouver \\
%14204 NE Salmon Creek Avenue \\
%Vancouver, Washington 98686-9600 \\
%\verb$dick_lang@vancouver.wsu.edu$ \\
%(360) 546-9632\\
%
%\pagebreak
%
%Orest Pilskalns\\
%Chief Executive Officer\\
%mapwith.us\\
%5512 NE 109th Court Suite J\\
%Vancouver, WA 98662 \\
%{\tt orest@mapwith.us} \\
%(360) 901-8813\\
%
%Sankar Jayaram\\  
%Co-Founder, President\\ % (Professor, MME WSU)\\
%%{\tt sjayaram@ie-sol.com} \\
%Uma Jayaram \\
%Co-Founder, EVP \\ %(Associate Professor, MME WSU)\\
%%{\tt ujayaram@ie-sol.com} \\
%{\tt uma.jayaram@intel.com} \\
%Voke VR / Intel \\
%3201 Scott Blvd, \\
%Santa Clara, CA, \\
%USA 95054 \\
%
%%3D-4U and Integrated Engineering Solutions (IES) \\
%%1610 NE Eastgate Blvd. \\
%%Suite 440 Pullman, WA 99163\\
%%(509) 335-6454\\
%
%%Roger C. Ray\\
%%#Principal Engineer, Intel (retired)\\
%%{\tt roger@anastasia.com} \\
%%(503) 292-1476.\\
% 
%John C. Hart\\
%Professor, Associate Dean of the Graduate College\\
%Department of Computer Science\\
%University of Illinois\\
%%3227 Siebel Center \\
%201 N. Goodwin \\
%Urbana, IL 61801 \\
%%{\tt jch@cs.uiuc.edu} \\
%{\tt jch@illinois.edu} \\
%(217) 333-8740 \\
%
%%Patrick J. Flynn, Associate Professor,
%%Department of Computer Science and Engineering,
%%University of Notre Dame,
%%384 Fitzpatrick Hall of Engineering,
%%Notre Dame, Indiana 46556,
%%(219) 631-8803.\\
%
%%Phil Smith, President,
%%Raster Graphics Inc,
%%1804 SE First Street,
%%Redmond, OR 97756,
%%(541) 923-5530.
%% now http://www.rastergraf.com/index.htm (7/8/2014)

\end{document}

